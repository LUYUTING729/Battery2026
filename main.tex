\documentclass{article}
\usepackage{graphicx} % Required for inserting images
\usepackage{amsmath}
\usepackage{ctex} 
\title{Battery swap}
\author{雨婷 陆}
\date{December 2025}

\begin{document}


\section{运营阶段:多中心车辆调度与路径优化模型(SCF 强模型)}
\label{sec:stage2_operational_scf}

本节构建运营阶段(Operational Stage)的精确优化模型。在战略阶段已确定开放的中心/充电站集合后,运营阶段需要决定:每辆车从哪个中心出发(可不出车),以及车辆的服务路径,使得所有客户被恰好服务一次,并满足车辆容量与续航(里程/能耗)限制,同时最小化路线行驶成本与车辆从初始位置调度至中心的切换成本。

\subsection{问题描述}
设系统中存在一组已开放中心(亦可理解为配送中心/充电站)以及一组客户需求点。每辆车在运营开始时位于一个初始位置,若车辆被启用,则必须被派遣至某一开放中心并从该中心出发执行配送任务,最终回到同一中心形成闭环路线。每个客户具有确定需求量,车辆具有统一容量上限与续航上限。运营决策包括:(i)车辆-中心派遣决策;(ii)车辆路径决策;(iii)满足容量与续航的可行性约束。目标是在保证服务覆盖与可行性的前提下,最小化系统的总成本。

\subsection{基本假设}
为使模型表达清晰且与实际运营场景一致,本文采用如下假设(可根据应用场景扩展):
\begin{enumerate}
    \item \textbf{中心集合已知。} 战略阶段已确定开放中心集合 $J$,运营阶段仅在这些中心上派遣车辆并构造路线。
    \item \textbf{单次服务、不可拆分需求。} 每个客户 $i$ 的需求 $q_i$ 必须由\emph{单辆车}一次性满足,不允许拆单(split delivery)。
    \item \textbf{车辆闭环运营。} 任一被启用车辆必须从某一中心 $j\in J$ 出发并回到同一中心结束;车辆不可在不同中心之间行驶(禁止中心到中心弧)。
    \item \textbf{容量与续航上限。} 车辆容量上限为 $U$,续航上限为 $Q$(可解释为最大可行驶距离或最大可消耗能量),均为确定性参数。
    \item \textbf{成本可加性与确定性。} 路径成本与调度成本均为确定性且可加:车辆行驶弧成本为 $c_{uv}$,车辆从初始位置 $p$ 调度到中心 $j$ 的成本为 $w_{pj}$。
    \item \textbf{无时间窗(本节)。} 本节不考虑时间窗与服务时间;若需扩展,可在续航势能变量基础上引入到达时间变量并添加时间窗约束。
\end{enumerate}

\subsection{集合与弧集}
\begin{itemize}
    \item $I$:客户集合,索引 $i$。
    \item $J$:开放中心(充电站/配送中心)集合,索引 $j$。
    \item $P$:车辆初始位置集合,索引 $p$。
    \item $\mathcal{K}$:车辆集合,索引 $k$。
    \item $V := I \cup J$:运营路径网络中的全部节点集合。
    \item $A \subseteq V\times V$:允许弧集。为符合闭环与中心运营机制,禁止自环与中心到中心弧:
    \[
    A := \{(u,v)\in V\times V:\ u\neq v,\ \neg(u\in J \wedge v\in J)\}.
    \]
\end{itemize}

\subsection{参数定义}
\begin{itemize}
    \item $q_i$:客户 $i\in I$ 的需求量。
    \item $U$:车辆容量上限。
    \item $Q$:车辆续航上限(最大距离/最大能耗)。
    \item $c_{uv}$:车辆在弧 $(u,v)\in A$ 上的行驶成本。
    \item $d_{uv}$:车辆在弧 $(u,v)\in A$ 上的行驶距离/能耗。
    \item $w_{pj}$:车辆从初始位置 $p\in P$ 调度至中心 $j\in J$ 的切换成本。
    \item $\mathrm{orig}(k)\in P$:车辆 $k$ 的初始位置。
    \item $K^{\max}_j$:中心 $j$ 最大可派出车辆数(车辆出车能力或充电口数量等)。
\end{itemize}

\subsection{决策变量定义}
\begin{itemize}
    \item $x_{uv}^k\in\{0,1\}$:若车辆 $k$ 行驶弧 $(u,v)\in A$ 则为 1,否则为 0。
    \item $z_{kj}\in\{0,1\}$:若车辆 $k$ 被派遣从中心 $j$ 出发(并回到该中心)则为 1,否则为 0。
    \item $f_{uv}^k\ge 0$:单商品流(Single-Commodity Flow, SCF)在弧 $(u,v)\in A$ 上的流量变量,用于强化容量与连通性(消除仅由客户构成的子回路)。
    \item $\tau_i^k\ge 0$:车辆 $k$ 抵达客户 $i$ 时的续航势能(累计距离/能耗)变量,用于保证路径总续航不超过 $Q$。
\end{itemize}

为便于表述,定义车辆 $k$ 是否服务客户 $i$ 的辅助表达:
\[
y_i^k := \sum_{(u,i)\in A} x_{ui}^k \in [0,1].
\]
在整数解中 $y_i^k\in\{0,1\}$,表示客户 $i$ 是否由车辆 $k$ 服务。

\subsection{目标函数}
\begin{equation}
\min \;
\underbrace{\sum_{k\in\mathcal{K}}\sum_{(u,v)\in A} c_{uv}\,x_{uv}^k}_{\text{路径行驶成本}}
+
\underbrace{\sum_{k\in\mathcal{K}}\sum_{j\in J} w_{\mathrm{orig}(k),j}\,z_{kj}}_{\text{车辆调度/切换成本}}.
\label{eq:obj_scf}
\end{equation}

\paragraph{解释.}
第一项为车辆在运营路网中行驶产生的可加成本;第二项刻画车辆从其初始位置切换到所选中心的调度成本,体现“车辆初始位置切换成本”的业务含义。

\subsection{约束条件}

\subsubsection{车辆派遣与服务覆盖约束}
\paragraph{(C1) 车辆最多选择一个中心.}
\begin{equation}
\sum_{j\in J} z_{kj} \le 1,\quad \forall k\in\mathcal{K}.
\label{eq:c1}
\end{equation}
\paragraph{解释.}
每辆车要么不出车(左侧为 0),要么从某一个中心出发(左侧为 1),避免同一车辆在多个中心同时出车。

\paragraph{(C2) 客户恰好服务一次.}
\begin{equation}
\sum_{k\in\mathcal{K}}\sum_{(u,i)\in A} x_{ui}^k = 1,\quad \forall i\in I.
\label{eq:c2}
\end{equation}
\paragraph{解释.}
每个客户节点在全系统范围内恰好被某辆车访问一次(用“入弧=1”表达),保证服务覆盖且不重复服务。

\paragraph{(C3) 客户点车辆流守恒(路径连续性).}
\begin{equation}
\sum_{(u,i)\in A} x_{ui}^k = \sum_{(i,v)\in A} x_{iv}^k,
\quad \forall i\in I,\ \forall k\in\mathcal{K}.
\label{eq:c3}
\end{equation}
\paragraph{解释.}
对每辆车而言,进入某客户则必须从该客户离开,保证路径连续且防止“某车进入、另一车离开”的非物理解。

\paragraph{(C4) 中心出入度与派遣决策一致(闭环路线).}
\begin{align}
\sum_{(j,v)\in A} x_{jv}^k &= z_{kj},\quad \forall j\in J,\ \forall k\in\mathcal{K},
\label{eq:c4a}\\
\sum_{(u,j)\in A} x_{uj}^k &= z_{kj},\quad \forall j\in J,\ \forall k\in\mathcal{K}.
\label{eq:c4b}
\end{align}
\paragraph{解释.}
若车辆 $k$ 从中心 $j$ 出车($z_{kj}=1$),则必须从 $j$ 出发一次并最终回到 $j$ 一次,形成闭环;若 $z_{kj}=0$,则车辆在 $j$ 的出入度均为 0,避免访问未派遣中心。

\paragraph{(C5) 中心车辆数量上限.}
\begin{equation}
\sum_{k\in\mathcal{K}} z_{kj} \le K^{\max}_j,\quad \forall j\in J.
\label{eq:c5}
\end{equation}
\paragraph{解释.}
每个中心可派遣车辆数受制于车队规模、充电设施数量或装载能力等运营限制。

\paragraph{(C6) 车辆启用联动.}
\begin{equation}
\sum_{(u,i)\in A} x_{ui}^k \le \sum_{j\in J} z_{kj},
\quad \forall i\in I,\ \forall k\in\mathcal{K}.
\label{eq:c6}
\end{equation}
\paragraph{解释.}
若车辆 $k$ 未被派遣(右侧为 0),则不能访问任何客户;该约束显式阻止“未派遣车辆服务客户”的不一致解。

\subsubsection{SCF 强化:容量与连通性约束}
本节采用单商品流(SCF)结构对容量与连通性进行联合建模。其核心思想是:对车辆 $k$,若其服务一组客户,则必须从所选中心向外提供足够的“流量”(可理解为总需求量),并在每个被服务客户处消耗对应需求。由此可在 LP 松弛层面有效消除仅由客户构成的子回路(subtour),从而显著增强下界。

\paragraph{(C7) 流量-弧选择联动.}
\begin{equation}
0 \le f_{uv}^k \le U\,x_{uv}^k,
\quad \forall (u,v)\in A,\ \forall k\in\mathcal{K}.
\label{eq:c7}
\end{equation}
\paragraph{解释.}
若弧未被选择($x_{uv}^k=0$),则流量必须为 0;若弧被选择,则流量不超过车辆容量 $U$,从而在流层面刻画容量上限。

\paragraph{(C8) 客户流守恒与需求消耗.}
\begin{equation}
\sum_{(u,i)\in A} f_{ui}^k - \sum_{(i,v)\in A} f_{iv}^k
=
q_i \sum_{(u,i)\in A} x_{ui}^k,
\quad \forall i\in I,\ \forall k\in\mathcal{K}.
\label{eq:c8}
\end{equation}
\paragraph{解释.}
若客户 $i$ 由车辆 $k$ 服务(右侧为 $q_i$),则在该客户处净消耗 $q_i$ 的流量;若不服务(右侧为 0),则流入等于流出。该结构确保流量只能从中心逐步传递至被服务客户。

\paragraph{(C9) 中心作为唯一源点(以派遣变量门控).}
定义车辆 $k$ 服务的总需求为
\[
D_k := \sum_{i\in I} q_i \sum_{(u,i)\in A} x_{ui}^k.
\]
对每个中心 $j$,要求车辆 $k$ 的流量净流出等于 $D_k$,但仅当车辆确实选择该中心($z_{kj}=1$)时生效:
\begin{align}
\sum_{(j,v)\in A} f_{jv}^k - \sum_{(u,j)\in A} f_{uj}^k
&\ge D_k - M_D(1-z_{kj}),
\quad \forall j\in J,\ \forall k\in\mathcal{K},
\label{eq:c9a}\\
\sum_{(j,v)\in A} f_{jv}^k - \sum_{(u,j)\in A} f_{uj}^k
&\le D_k + M_D(1-z_{kj}),
\quad \forall j\in J,\ \forall k\in\mathcal{K}.
\label{eq:c9b}
\end{align}
其中 $M_D$ 为足够大的常数。取 $M_D=\sum_{i\in I} q_i$ 可给出紧的上界。

\paragraph{解释.}
当 $z_{kj}=1$ 时,上式迫使中心 $j$ 为车辆 $k$ 提供恰好 $D_k$ 的净输出流,从而成为车辆路径连通结构的“源”。若存在一个仅由客户构成的子回路,则该子回路内客户存在正的需求消耗,但没有任何中心提供源流,因而在该结构下(即使在 LP 松弛中)也无法满足约束,从而有效消除子回路并增强线性松弛界。

\subsubsection{续航约束(势能法,带门控)}
续航约束要求车辆从中心出发至服务客户并回到中心的总距离/能耗不超过 $Q$。为此定义客户处势能变量 $\tau_i^k$,并在弧选择时进行势能递推。

\paragraph{(C10) 势能上界(仅对被访问客户生效).}
\begin{equation}
0 \le \tau_i^k \le Q \sum_{(u,i)\in A} x_{ui}^k,
\quad \forall i\in I,\ \forall k\in\mathcal{K}.
\label{eq:c10}
\end{equation}
\paragraph{解释.}
若车辆 $k$ 不访问客户 $i$,则势能被压为 0;若访问,则势能不超过 $Q$。

\paragraph{(C11) 客户--客户势能递推.}
\begin{equation}
\tau_h^k \ge \tau_i^k + d_{ih} - Q(1-x_{ih}^k),
\quad \forall k\in\mathcal{K},\ \forall i\neq h\in I.
\label{eq:c11}
\end{equation}
\paragraph{解释.}
当弧 $(i,h)$ 被车辆 $k$ 选择时($x_{ih}^k=1$),该约束强制 $\tau$ 累积增加至少 $d_{ih}$;当弧未选择时,通过大 $M$(此处取 $Q$)放松。

\paragraph{(C12) 中心--客户初始化(以 $z_{kj}$ 门控).}
\begin{equation}
\tau_i^k \ge d_{ji} - Q(1-x_{ji}^k) - M_\tau(1-z_{kj}),
\quad \forall k\in\mathcal{K},\ \forall j\in J,\ \forall i\in I.
\label{eq:c12}
\end{equation}
\paragraph{解释.}
当车辆 $k$ 选择中心 $j$ 且从 $j$ 直接到达客户 $i$ 时,初始化势能不小于 $d_{ji}$。门控项 $(1-z_{kj})$ 确保未被车辆选择的中心不会对 $\tau$ 施加不必要下界(这是避免错误不可行与弱松弛的关键)。

\paragraph{(C13) 回中心可行性(以 $z_{kj}$ 门控).}
\begin{equation}
\tau_i^k + d_{ij}
\le
Q + M_\tau(1-x_{ij}^k) + M_\tau(1-z_{kj}),
\quad \forall k\in\mathcal{K},\ \forall j\in J,\ \forall i\in I.
\label{eq:c13}
\end{equation}
\paragraph{解释.}
若车辆 $k$ 选择中心 $j$ 且从客户 $i$ 返回 $j$,则要求到达客户 $i$ 的累计势能加上返程距离不超过 $Q$。门控项保证仅对车辆真实选择的中心生效。

\paragraph{大 $M$ 取值说明.}
对于 \eqref{eq:c9a}--\eqref{eq:c9b},取 $M_D=\sum_{i\in I}q_i$ 为紧的选择;对于 \eqref{eq:c12}--\eqref{eq:c13},可取 $M_\tau=Q$(安全且较紧)。

\subsection{模型小结与性质说明}
\begin{itemize}
    \item \textbf{等价性:} 约束 \eqref{eq:c1}--\eqref{eq:c6} 保证车辆派遣机制与闭环路径结构;\eqref{eq:c7}--\eqref{eq:c9b} 保证容量可行并消除客户子回路;\eqref{eq:c10}--\eqref{eq:c13} 保证续航上限。整体上与“多中心车辆路径问题(MDVRP)+ 车辆派遣 + 容量与续航限制”的业务语义等价。
    \item \textbf{LP 强度:} 与 MTZ 类子回路消除相比,SCF 在 LP 松弛层面即可排除大规模分数子回路,从而通常带来显著更紧的下界与更快的 incumbent 搜索速度。
    \item \textbf{可扩展性:} 若需引入时间窗、服务时间、异质车辆或充电补能过程,可在现有结构上继续添加到达时间变量及其传播约束,或对不同车辆设定 $U_k, Q_k$。
\end{itemize}

\section{基于集合划分的列生成建模框架}
\label{sec:sp_cg_model}

本节将运营阶段的弧变量 SCF 强模型重构为基于集合划分(Set Partitioning, SP)的列生成建模框架。该框架以“可行车辆闭环路线”为基本决策单元,将指数规模的路径组合空间隐式表示在列集合中,并通过定价子问题按需生成具有负化简成本的路线列。

该建模方式是车辆路径问题(VRP)精确算法中的主流范式,能够显著增强线性松弛强度,并为后续分支定价或分支割定价算法奠定基础。

\subsection{路线列定义}

设车辆集合为 $\mathcal{K}$,中心集合为 $J$,客户集合为 $I$。对任意车辆 $k\in\mathcal{K}$ 与中心 $j\in J$,定义其可行闭环路线集合为:
\[
\mathcal{R}_{kj} :=
\left\{
r:\ j \to i_1 \to i_2 \to \dots \to i_m \to j
\;\middle|\;
\sum_{i\in I} q_i a_{ir} \le U,\;
\sum_{(u,v)\in r} d_{uv} \le Q,\;
\text{且客户不重复访问}
\right\}.
\]

其中:
\begin{itemize}
    \item $a_{ir}\in\{0,1\}$:若路线 $r$ 服务客户 $i$,则为1;
    \item $\sum_i q_i a_{ir} \le U$:容量约束;
    \item $\sum_{(u,v)\in r} d_{uv} \le Q$:续航(路线长度)约束。
\end{itemize}

\paragraph{列变量.}
对每条路线 $r\in\mathcal{R}_{kj}$,定义决策变量:
\[
\lambda_{kjr}\in\{0,1\},
\]
表示车辆 $k$ 是否选择从中心 $j$ 出发执行路线 $r$。

\paragraph{列成本.}
路线列的总成本由两部分组成:
\[
c_{kjr}
=
\sum_{(u,v)\in r} c_{uv}
+
w_{\mathrm{orig}(k),j},
\]
其中第一项为路线行驶成本,第二项为车辆从初始位置切换至中心 $j$ 的调度成本。

\subsection{集合划分主问题(RMP)}

基于上述路线列定义,可构建集合划分主问题如下:

\begin{align}
\min \quad
& \sum_{k\in\mathcal{K}}\sum_{j\in J}\sum_{r\in\mathcal{R}_{kj}}
c_{kjr}\lambda_{kjr}
\label{eq:sp_obj}
\\
\text{s.t.}\quad
& \sum_{k\in\mathcal{K}}\sum_{j\in J}\sum_{r\in\mathcal{R}_{kj}}
a_{ir}\lambda_{kjr}
=1,
&& \forall i\in I
\label{eq:sp_cover}
\\
& \sum_{j\in J}\sum_{r\in\mathcal{R}_{kj}}
\lambda_{kjr}
\le 1,
&& \forall k\in\mathcal{K}
\label{eq:sp_vehicle}
\\
& \sum_{k\in\mathcal{K}}\sum_{r\in\mathcal{R}_{kj}}
\lambda_{kjr}
\le K_j^{\max},
&& \forall j\in J
\label{eq:sp_depot}
\\
& \lambda_{kjr}\in\{0,1\},
&& \forall k,j,r.
\label{eq:sp_binary}
\end{align}

\paragraph{约束解释.}
\begin{itemize}
    \item \eqref{eq:sp_cover}:每个客户恰好被一条路线服务一次;
    \item \eqref{eq:sp_vehicle}:每辆车至多执行一条路线;
    \item \eqref{eq:sp_depot}:每个中心的出车数量不超过其能力上限。
\end{itemize}

由于每条列本身已保证“从中心出发并回到该中心”的闭环结构,因此无需在主问题中再显式建模路径连续性或中心出入度约束。

\subsection{主问题的对偶与化简成本}

在列生成过程中,首先对主问题的线性松弛($\lambda\ge 0$)求解,并引入如下对偶变量:

\begin{itemize}
    \item $\pi_i$:客户覆盖约束 \eqref{eq:sp_cover} 的对偶变量;
    \item $\alpha_k\ge 0$:车辆使用约束 \eqref{eq:sp_vehicle} 的对偶变量;
    \item $\beta_j\ge 0$:中心容量约束 \eqref{eq:sp_depot} 的对偶变量。
\end{itemize}

则任意列 $(k,j,r)$ 的化简成本(reduced cost)为:
\begin{equation}
\bar c_{kjr}
=
c_{kjr}
-
\sum_{i\in I}\pi_i a_{ir}
-
\alpha_k
-
\beta_j.
\label{eq:reduced_cost}
\end{equation}

若存在 $\bar c_{kjr}<0$ 的列,则将其加入主问题;若所有列的化简成本均非负,则当前主问题线性松弛已达到最优。

\subsection{定价子问题:RCSPP 建模}

对固定车辆 $k$ 与中心 $j$,定价子问题的目标是寻找一条从 $j$ 出发回到 $j$ 的可行路线 $r$,使得其化简成本最小。

由于 $-\alpha_k-\beta_j$ 为常数项,定价子问题可等价写为:

\begin{align}
\min_{r}
\quad
& \sum_{(u,v)\in r} c_{uv}
-
\sum_{i\in I} \pi_i a_{ir}
\label{eq:pricing_obj}
\\
\text{s.t.}\quad
& \sum_{i\in I} q_i a_{ir} \le U
\label{eq:pricing_cap}
\\
& \sum_{(u,v)\in r} d_{uv} \le Q
\label{eq:pricing_range}
\\
& r:\ j\rightarrow j,\quad
\text{且客户至多访问一次}.
\end{align}

该问题即为带资源约束的最短路径问题(Resource Constrained Shortest Path Problem, RCSPP);若要求客户不重复访问,则为元素路径版本(ESPPRC)。

\subsection{ng-route 放松的定价子问题}

为降低 ESPPRC 的计算复杂度,本研究在定价阶段采用 ng-route 放松。其核心思想是:

\begin{itemize}
    \item 对每个客户 $i$ 定义一个邻域集合 $N_i$;
    \item 在路径扩展过程中,仅在最近邻集合内强制元素性约束;
    \item 允许在远离该邻域的节点发生受控重复访问,从而减少状态空间。
\end{itemize}

在标签扩展过程中,每个标签 $L$ 记录:
\[
L = (i,\; \text{cost},\; \text{load},\; \text{dist},\; \mathcal{V}_L),
\]
其中:
\begin{itemize}
    \item $i$:当前节点;
    \item $\text{cost}$:累计化简成本;
    \item $\text{load}$:累计载重;
    \item $\text{dist}$:累计距离;
    \item $\mathcal{V}_L$:ng-route 访问记忆集合。
\end{itemize}

标签扩展需满足:
\[
\text{load} \le U,\qquad
\text{dist} \le Q,
\]
且若扩展至节点 $h\in \mathcal{V}_L$,则该扩展被禁止。

通过该放松,定价子问题仍保持 RCSPP 的结构,但显著减少状态数量,在实践中可获得接近 ESPPRC 的下界质量。

\subsection{模型结构总结}

通过上述变换,原始弧变量 SCF 模型被等价重构为:

\begin{itemize}
    \item 主问题:集合划分模型,决定路线选择与车辆派遣;
    \item 子问题:以 RCSPP(ng-route 放松)形式求解的定价问题;
    \item 整体算法:列生成或分支定价框架。
\end{itemize}

该结构避免了弧变量模型中的大规模流变量与大 $M$ 门控约束,使得线性松弛更紧,并为大规模实例的精确求解提供了可扩展的算法基础。

\section{Cuts 与定价加速:面向 MDVRP--RL 的鲁棒分支割定价设计}
\label{sec:cuts_pricing_accel}

本节系统总结可用于加速本文 \emph{多中心有限车队车辆路径问题}(MDVRP,含容量约束与路线长度/续航约束)的割平面(cuts)与定价(pricing)加速策略。我们采用集合划分(Set Partitioning, SP)主问题与资源约束最短路(RCSPP)定价子问题的标准分支割定价(Branch-Cut-and-Price, BCP)范式\cite{Lubbecke2005,Pessoa2008,Feillet2010}。本文特别关注两类技术要点:
(i) \emph{鲁棒(robust)割}:不破坏定价子问题的伪多项式可解性;
(ii) \emph{定价强化与剪枝}:在 ng-route 放松与资源双维(载重、距离)约束下显著降低状态空间与定价次数。

\subsection{问题与符号回顾:SP 主问题与定价结构}
\label{subsec:recap_sp_pricing}

设客户集合为 $I$,中心集合为 $J$,车辆集合为 $\mathcal{K}$。对任意车辆--中心对 $(k,j)$,记可行闭环路线集合为 $\mathcal{R}_{kj}$,并以 $\lambda_{kjr}\in\{0,1\}$ 表示是否选择路线列 $r\in\mathcal{R}_{kj}$。
令 $a_{ir}\in\{0,1\}$ 表示路线 $r$ 是否服务客户 $i$,则 SP 主问题(RMP)的核心覆盖约束为
\begin{equation}
\sum_{k\in\mathcal{K}}\sum_{j\in J}\sum_{r\in\mathcal{R}_{kj}} a_{ir}\lambda_{kjr}=1,\quad \forall i\in I.
\label{eq:cover}
\end{equation}
此外存在车辆使用上限与中心出车上限约束(略)。

在列生成中,对覆盖约束 \eqref{eq:cover} 引入对偶变量 $\pi_i$,对车辆与中心上限引入对偶 $\alpha_k,\beta_j$,则列 $(k,j,r)$ 的化简成本为
\begin{equation}
\bar c_{kjr}=c_{kjr}-\sum_{i\in I}\pi_i a_{ir}-\alpha_k-\beta_j,
\label{eq:rc_master}
\end{equation}
其中 $c_{kjr}$ 为路线行驶成本与车辆切换成本之和。定价子问题对固定 $(k,j)$ 等价为在图 $G=(V,A)$ 上寻找从 $j$ 出发回到 $j$ 的闭环路径 $r$ 使得
\begin{equation}
\min\;\sum_{(u,v)\in r} c_{uv}-\sum_{i\in I}\pi_i a_{ir}
\quad \text{s.t.}\quad
\sum_{i\in I}q_i a_{ir}\le U,\;\sum_{(u,v)\in r} d_{uv}\le Q,
\label{eq:pricing_rcspp}
\end{equation}
该问题为 RCSPP;若要求元素路径则为 ESPPRC\cite{Feillet2010}。本文假设 $d_{uv}$ 与 $c_{uv}$ 同度量(例如均为路程),从而可在剪枝与支配规则中获得更强结构性优势(见 \S\ref{subsec:dominance})。

\subsection{主问题割平面:鲁棒 cuts 与非鲁棒 cuts}
\label{subsec:master_cuts}

在 BCP 框架下,主问题割平面通常直接施加于 $\lambda$ 变量空间,用以收紧 LP 松弛并减少分支树规模。根据割对定价结构的影响,可将其分为 \emph{鲁棒割} 与 \emph{非鲁棒割}\cite{Pessoa2008}。

\subsubsection{Rounded Capacity Inequalities(RCC):容量割(鲁棒、首选)}
\label{subsubsec:rcc}

\paragraph{动机与形式.}
容量约束 $\sum_i q_i a_{ir}\le U$ 蕴含:任意客户子集 $S\subseteq I$ 的总需求量 $q(S)=\sum_{i\in S}q_i$ 至少需要
\[
\eta(S):=\left\lceil \frac{q(S)}{U}\right\rceil
\]
条车辆路线来服务。由此可得到经典的 \emph{rounded capacity cut}(RCC)家族\cite{Lysgaard2004CVRPSEP}:
\begin{equation}
\sum_{k\in\mathcal{K}}\sum_{j\in J}\sum_{r\in\mathcal{R}_{kj}}
\delta(S,r)\,\lambda_{kjr}
\;\ge\;
\eta(S),
\quad \forall S\subseteq I,
\label{eq:rcc}
\end{equation}
其中 $\delta(S,r)\in\{0,1\}$ 表示路线 $r$ 是否访问集合 $S$(至少包含一个客户于 $S$)。

\paragraph{有效性推导(证明思路).}
对任意可行整数解,令 $\mathcal{T}$ 表示被选中的路线集合。由于每条路线 $r\in\mathcal{T}$ 的载重至多为 $U$,这些路线用于服务 $S$ 的总可交付需求不超过 $U\cdot |\{r\in\mathcal{T}:\delta(S,r)=1\}|$。另一方面,覆盖约束 \eqref{eq:cover} 要求 $S$ 内每个客户需求被某条路线承载,故必须满足
\[
U\cdot |\{r\in\mathcal{T}:\delta(S,r)=1\}|
\ge q(S).
\]
将两边除以 $U$ 并向上取整即得
\[
|\{r\in\mathcal{T}:\delta(S,r)=1\}|
\ge \left\lceil \frac{q(S)}{U}\right\rceil=\eta(S),
\]
等价于 \eqref{eq:rcc}。证毕。

\paragraph{分离与工程实现.}
RCC 的精确分离可转化为最小割/子集搜索问题,工业实现通常采用成熟的分离器(例如 CVRPSEP)或启发式分离\cite{Lysgaard2004CVRPSEP}。RCC 属于鲁棒 cuts:其加入不会改变定价子问题的 RCSPP 结构,仅通过对偶反馈改变 $\pi$ 的取值与负化简成本列的产生顺序\cite{Pessoa2008}。

\subsubsection{Clique cuts:基于不相容关系的团不等式(鲁棒、强)}
\label{subsubsec:clique}

\paragraph{不相容图构造.}
定义客户对 $(i,h)$ 的 \emph{不相容}:若任何可行闭环路线均无法同时服务 $i$ 与 $h$,则在冲突图 $G^{conf}$ 中加入边 $(i,h)$。在本文问题中,可由以下两类充分条件构造不相容边:
\begin{enumerate}
\item \textbf{容量不相容:} 若 $q_i+q_h>U$,则 $i$ 与 $h$ 不可同一路线;
\item \textbf{续航不相容(同度量优势):} 若对任意中心 $j\in J$,存在闭环长度下界
\[
d_{ji}+d_{ih}+d_{hj} > Q,
\]
则任何从 $j$ 出发回 $j$ 且访问 $\{i,h\}$ 的路线均违反续航约束,因此 $i,h$ 不相容。
\end{enumerate}

\paragraph{团不等式形式.}
对冲突图任意团(clique)$C\subseteq I$,任一可行路线最多服务团中一个客户,因此得到 clique cut\cite{Baldacci2008}:
\begin{equation}
\sum_{k\in\mathcal{K}}\sum_{j\in J}\sum_{r\in\mathcal{R}_{kj}}
\left(\sum_{i\in C} a_{ir}\right)\lambda_{kjr}
\;\le\; 1.
\label{eq:clique_cut}
\end{equation}

\paragraph{有效性证明.}
对任意可行整数解,设存在一条被选路线 $r$ 同时服务 $C$ 中两个不同客户 $i\neq h$。由于 $C$ 为团,$(i,h)$ 不相容,矛盾。因此任意被选路线至多包含 $C$ 中一个客户,故所有被选路线对团客户的覆盖计数之和不超过 1,即 \eqref{eq:clique_cut} 成立。证毕。

\paragraph{鲁棒性与实践意义.}
clique cuts 对 SP 松弛非常有效,且仍保持鲁棒性:其对偶信息可被吸收为客户访问的附加惩罚,不要求定价显式追踪复杂组合结构\cite{Pessoa2008,Baldacci2008}。

\subsubsection{Subset Row Inequalities(SRI):秩1 Chv\'atal--Gomory 割(鲁棒)}
\label{subsubsec:sri}

\paragraph{背景.}
SRI 是针对集合划分/覆盖矩阵的经典秩1 CG-cut,在 VRPTW/VRP 的 BCP 中常被用于打击分数覆盖结构\cite{Jepsen2008}。

\paragraph{构造(概念性推导).}
考虑覆盖等式 \eqref{eq:cover} 的一个行子集 $\mathcal{S}\subseteq I$,对这些行作线性组合并进行 CG 向上取整,可得到形如
\begin{equation}
\sum_{k,j,r} \left\lceil \sum_{i\in \mathcal{S}}\theta_i a_{ir}\right\rceil \lambda_{kjr}
\;\ge\; \left\lceil \sum_{i\in\mathcal{S}}\theta_i \right\rceil,
\label{eq:sri_template}
\end{equation}
其中 $\theta_i\in(0,1]$ 为组合系数。实际实现中,SRI 的分离通常通过启发式或特定结构的最违反行组合获得\cite{Jepsen2008}。

\paragraph{鲁棒性.}
SRI 作为主问题的秩1 cut 通常被视为鲁棒 cuts 的一类:其不会要求定价携带难以处理的全局组合记忆,而只通过对偶改变列的相对吸引力\cite{Pessoa2008,Jepsen2008}。

\subsubsection{非鲁棒 cuts:使用原则与适用场景}
\label{subsubsec:nonrobust}

\paragraph{定义与风险.}
非鲁棒 cuts 指加入后会破坏定价子问题的可解结构(例如迫使定价携带高维全局计数/组合状态,导致伪多项式 DP 失效),或使得 ``pricing-as-RCSPP'' 不再成立\cite{Pessoa2008}。

\paragraph{实践建议.}
对于本文问题,我们建议将非鲁棒 cuts 控制在如下场景:
\begin{itemize}
\item \textbf{根节点少量加入:} 仅添加最违反的一小批,以快速压缩 root gap;
\item \textbf{后期整数化:} 当列集合已较丰富、定价调用次数下降时,可适当使用更强但更重的 cuts;
\item \textbf{与 MILP 定价配合:} 若某些节点采用 MILP 求解定价(见 \S\ref{subsec:pricing_mip}),则可承受非鲁棒 cuts 带来的结构破坏。
\end{itemize}

\subsection{定价子问题强化与加速:ng-route、支配规则与边剪枝}
\label{subsec:pricing_accel}

在本文框架中,定价子问题为带两类资源(载重、距离)的 RCSPP。直接求解 ESPPRC 通常代价高昂,因此我们采用 ng-route 放松并辅以支配规则、$k$-cycle 消除以及基于下界的边剪枝\cite{Feillet2010,IrnichVilleneuve2006,Contardo2014}。

\subsubsection{ng-route 放松:受控非元素路径与正确性}
\label{subsubsec:ngroute}

\paragraph{定义.}
对每个客户 $i$ 给定邻域集合 $N_i\subseteq I$(如按距离最近的 $p$ 个客户)。在标签扩展中,维护访问记忆集合 $\mathcal{V}$,其更新规则可写为
\begin{equation}
\mathcal{V}' \;=\; \left(\mathcal{V}\cap N_h\right)\cup\{h\},
\label{eq:ng_update}
\end{equation}
其中 $h$ 为扩展到的新节点。若 $h\in\mathcal{V}$ 则禁止扩展,以此在局部邻域上强制元素性。

\paragraph{性质(放松的下界有效性).}
ng-route 允许一定程度的远邻重复访问,因此其可行域包含 ESPPRC 的可行域(即对元素性要求的放松),从而定价得到的最优值不大于 ESPPRC 的最优值,进而主问题 LP 下界不会被破坏(但可能变弱)。这一 ``可行域包含'' 关系与其对 CG 下界的影响可参考非元素路径定价策略的系统讨论\cite{Feillet2010,Martinelli2014}。

\subsubsection{$k$-cycle elimination:小环消除(鲁棒强化)}
\label{subsubsec:kcycle}

\paragraph{动机.}
ng-route 仍可能产生包含短环的非元素路径,造成大量 ``伪负化简成本'' 列。为此可加入 $k$-cycle elimination(典型 $k=2,3$)以排除短环结构\cite{IrnichVilleneuve2006}。

\paragraph{有效性说明(以 2-cycle 为例).}
2-cycle 指路径片段 $i\to h\to i$。由于 $d=c$ 同度量且边权非负,该片段在不增加覆盖价值的前提下增加成本与资源消耗,因此在最优定价路径中通常可被支配排除。将其作为显式禁忌可减少状态空间并提升定价效率\cite{IrnichVilleneuve2006}。

\subsubsection{支配(dominance)规则与证明:双资源 RCSPP 的标准剪枝}
\label{subsec:dominance}

定价采用标签法(label-setting/label-correcting)时,核心在于支配规则以删去无必要标签\cite{Feillet2010}。

\paragraph{标签定义.}
每个标签 $L$ 表示从起点中心 $j$ 到当前节点 $v$ 的一条部分路径,其状态包括:
\[
L=(v,\; \mathrm{rc}(L),\; \mathrm{load}(L),\; \mathrm{dist}(L),\; \mathcal{V}(L)),
\]
其中 $\mathrm{rc}(L)$ 为累计 ``定价目标''(行驶成本减对偶收益),$\mathrm{load}$ 与 $\mathrm{dist}$ 分别为载重与距离资源消耗,$\mathcal{V}$ 为 ng-memory。

\paragraph{引理(支配充分条件).}
\begin{theorem}[标签支配]
\label{thm:dominance}
考虑到达同一节点 $v$ 的两个标签 $L_1,L_2$。若满足
\begin{align}
&\mathrm{rc}(L_1)\le \mathrm{rc}(L_2),\quad
\mathrm{load}(L_1)\le \mathrm{load}(L_2),\quad
\mathrm{dist}(L_1)\le \mathrm{dist}(L_2),\label{eq:dom_res}\\
&\mathcal{V}(L_1)\subseteq \mathcal{V}(L_2),\label{eq:dom_mem}
\end{align}
且至少一个不等式严格成立,则 $L_1$ 支配 $L_2$,即以 $L_2$ 为前缀的任何可行扩展都可由 $L_1$ 的扩展构造出不更差的可行解,因此 $L_2$ 可安全删除。
\end{theorem}

\begin{proof}
任取 $L_2$ 的任意可行扩展序列(后缀路径)$P$。由于 \eqref{eq:dom_mem},$L_1$ 的 ng-memory 不比 $L_2$ 更严格,故 $P$ 中每次 ``禁止访问'' 判断在 $L_1$ 下仍可行(不会因为更大的记忆集合而被禁)。同时由 \eqref{eq:dom_res},$L_1$ 的资源消耗不超过 $L_2$,因此在载重与距离上,$L_1$ 扩展后仍满足资源上界。由于目标函数在扩展过程中对每条边的代价增量相同,且 $L_1$ 的初始累计目标不大于 $L_2$,故 $L_1$ 与相同后缀 $P$ 组合所得完整路径目标值不大于 $L_2$ 对应的完整路径目标值。由此 $L_2$ 不可能产生严格优于 $L_1$ 的解,可安全删除。证毕。
\end{proof}

\paragraph{同度量优势.}
当 $d=c$ 同度量且边权满足度量性质时,可加入更强的 \emph{completion bound}(例如从当前节点返回中心的最短路下界)并与 \eqref{thm:dominance} 结合得到更强剪枝;该思想属于 RCSPP 定价的标准增强\cite{Feillet2010}。

\subsubsection{基于下界的边剪枝(edge filtering):与 SCF 下界耦合的加速机制}
\label{subsubsec:edge_filtering}

\paragraph{思想.}
利用强下界(例如 vehicle-flow/SCF 松弛)对定价图进行弧删减:若某条弧 $(u,v)$ 在任何可能的负化简成本路线中都不可能出现,则可提前删除,从而缩小定价图规模并减少标签扩展次数\cite{Contardo2014}。

\paragraph{命题(充分条件).}
记某次定价迭代中,对固定中心 $j$,令 $\ell_{uv}$ 表示在定价目标下经过弧 $(u,v)$ 的最小可能 ``局部增量下界''(可由最短路预处理获得)。若对任意可行前缀到达 $u$ 的标签 $L$,均有
\begin{equation}
\mathrm{rc}(L)+\ell_{uv}+\mathrm{LB}_{v\to j} \ge 0,
\label{eq:edge_elim_cond}
\end{equation}
其中 $\mathrm{LB}_{v\to j}$ 为从 $v$ 返回中心 $j$ 的目标下界(忽略资源或用资源松弛下界),则弧 $(u,v)$ 不可能出现在任何负化简成本闭环路线中,可从定价图中删除。

\begin{proof}
对任意包含弧 $(u,v)$ 的闭环路线,其目标值可分解为:到达 $u$ 的前缀目标 $\mathrm{rc}(L)$,经过弧 $(u,v)$ 的增量,以及从 $v$ 返回中心的后缀目标。由下界定义,完整闭环目标值至少为 $\mathrm{rc}(L)+\ell_{uv}+\mathrm{LB}_{v\to j}$。若对所有可行前缀均满足 \eqref{eq:edge_elim_cond},则任何包含 $(u,v)$ 的闭环路线目标值均非负,因而其化简成本亦非负,不可能生成改善列。故该弧可安全删除。证毕。
\end{proof}

该类 ``用下界做 edge elimination'' 的有效性与工程价值在 MDVRP(capacity+route length) 的精确算法中已有清晰展示\cite{Contardo2014}。

\subsubsection{当定价改用 MILP(或混合定价)时的子问题有效不等式}
\label{subsec:pricing_mip}

在分支深处或资源维数增加时,label-setting 可能出现状态爆炸。此时可采用混合策略:先用启发式/标签法寻找负列,必要时用 MILP 精确求解定价并加入子问题层面的有效不等式(如元素性强化、资源逻辑割等)。此类 ``子问题 branch-and-cut'' 思路在 RCSPP/ESPPRC 研究中已有系统探索\cite{JepsenTechReport2008}。

\subsection{稳定化(Stabilization):减少对偶振荡与冗余定价}
\label{subsec:stabilization}

车辆路径问题的列生成常出现对偶振荡,导致定价重复生成相似列、收敛缓慢。稳定化方法通过限制对偶变量的变化范围或添加惩罚项,使列生成迭代更平滑\cite{Lubbecke2005,Pessoa2008}。常见做法包括:
\begin{itemize}
\item \textbf{Box stabilization:} 将对偶变量限制在区间 $[\underline\pi,\overline\pi]$;
\item \textbf{Penalty stabilization:} 在 RMP 目标中加入对偶偏离参考点的惩罚;
\item \textbf{Hybrid stabilization:} 组合多种稳定化策略并自适应调参\cite{Pessoa2008}.
\end{itemize}
稳定化通常不改变定价子问题形式,因此属于鲁棒的整体加速机制。

\subsection{小结:推荐的鲁棒 cut/加速组合(可用于消融实验)}
\label{subsec:summary_reco}

基于上述讨论,针对本文问题结构,推荐的 ``鲁棒优先'' 组合为:
\begin{enumerate}
\item 主问题:RCC(容量割) + clique cuts(由容量与续航不相容构造) + SRI(秩1割);
\item 定价:ng-route + $k$-cycle elimination + 强支配规则(含 completion bound);
\item 跨层耦合:SCF/vehicle-flow 下界驱动的 edge filtering;
\item 全局:列生成稳定化(box/penalty/hybrid)。
\end{enumerate}
该组合在不破坏 RCSPP 可解结构的前提下,通常能够显著压缩 root gap 并减少定价调用次数,从而提升整体 BCP 可扩展性\cite{Pessoa2008,Feillet2010,Contardo2014}。

\begin{thebibliography}{99}

\bibitem{Lubbecke2005}
M.~E. L{\"u}bbecke and J.~Desrosiers.
\newblock Selected topics in column generation.
\newblock \emph{Operations Research}, 53(6):1007--1023, 2005.

\bibitem{Pessoa2008}
A.~Pessoa, M.~Poggi de Arag{\~a}o, and others.
\newblock Branch-cut-and-price: a survey.
\newblock (Survey/Chapter version circulated on Optimization Online), 2008.

\bibitem{Feillet2010}
D.~Feillet.
\newblock A tutorial on column generation and branch-and-price for vehicle routing problems.
\newblock \emph{4OR}, 8:407--424, 2010.

\bibitem{Baldacci2008}
R.~Baldacci, N.~Christofides, and A.~Mingozzi.
\newblock An exact algorithm for the vehicle routing problem based on the set partitioning formulation with additional cuts.
\newblock \emph{Mathematical Programming}, 115:351--385, 2008.

\bibitem{Jepsen2008}
M.~Jepsen, B.~Petersen, S.~Spoorendonk, and S.~Ropke.
\newblock Subset-row inequalities applied to the vehicle-routing problem with time windows.
\newblock \emph{Operations Research}, 56(2):497--511, 2008.

\bibitem{Lysgaard2004CVRPSEP}
J.~Lysgaard, A.~N. Letchford, and R.~Eglese.
\newblock A new branch-and-cut algorithm for the capacitated vehicle routing problem.
\newblock (and the CVRPSEP separation framework; widely used implementation), 2004.

\bibitem{IrnichVilleneuve2006}
S.~Irnich and D.~Villeneuve.
\newblock The shortest path problem with resource constraints and $k$-cycle elimination for vehicle routing.
\newblock \emph{INFORMS Journal on Computing}, 18(4):391--406, 2006.

\bibitem{Contardo2014}
C.~Contardo.
\newblock The multi-depot vehicle routing problem with capacity and route length constraints.
\newblock \emph{EURO Journal on Transportation and Logistics}, 3:123--146, 2014.

\bibitem{Martinelli2014}
R.~Martinelli, R.~Mansini, and others.
\newblock New route relaxations and pricing strategies for the vehicle routing problem.
\newblock \emph{European Journal of Operational Research}, 2014.

\bibitem{JepsenTechReport2008}
M.~Jepsen.
\newblock Branch-and-cut algorithms for the elementary shortest path problem with resource constraints (and related inequalities).
\newblock Technical report, 2008.

\end{thebibliography}

\end{document}
